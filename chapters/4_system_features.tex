\chapter{خصوصیات سامانه}
\clearpage

\noindent {
این قسمت شامل جزییات فنی سیستم از قبیل نیازمند‌های عملیاتی، نیازمندی‌های غیر عملیاتی، نمودار‌های
\lr{UML}
\LTRfootnote{\lr{Unified Modeling Language}}
فرایند توسعه
\LTRfootnote{\lr{Development Process}}
دایکر و چرخه عرضه
\LTRfootnote{\lr{Release Cycle}}
این نرم‌افزار است.
}

\section {نیازمندی‌های عملیاتی}
\noindent {
در این قسمت نیازمندی‌های عملیاتی نرم‌افزار دایکر آورده شده اند.
}

\subsection {نیازمندی شماره ۱}
\subsubsection {اتصال به پایگاه‌های داده مختلف}
\noindent {
دایکر باید بتواند بدون مشکل به سیستم‌های مدیریت پایگاه‌های داده‌ی مختلف متصل شود(برای تامین داده‌ی لازم برای فعالیت‌هایش).
حداقل سیستم‌هایی که دایکر باید به آن‌ها متصل شود عبارند از
\lr{IBM DB2, Oracle DB} و
\lr{Microsoft SQL Server}.
}

\subsection {نیازمندی شماره ۲}
\subsubsection {\lr{Import} کردن اطلاعات مشتریان}
\noindent {
کاربر باید امکان
\lr{Import}
رکورد‌ها از پایگاه‌های داده‌ی مختلف را داشته باشد.
}

\subsection {نیازمندی شماره ۳}
\subsubsection {ثبت و مدیریت کاربران}
\noindent {
کاربری به نام مدیر(\lr{Admin}) باید بتواند کاربران مختلف را سیستم اضافه کند.
هر کاربر بسته به سطح دسترسی که او داده می‌شود می‌تواند از رکورد‌هایی که دایکر به آن‌ها دسترسی دارد
گزارش بگیرد.
}

\subsection {نیازمندی شماره ۴}
\subsubsection {صفحه \lr{Login} کاربران}
\noindent {
دایکر در ابتدای امر پس از بالا آمدن باید یک صفحه لاگین به کاربر نشان دهد که
سطح کاربری هر فرد را مشخص می‌کند.
}

\subsection {نیازمندی شماره ۵}
\subsubsection {\lr{Password Recovery}}
\noindent {
کاربران باید در صورت فراموش کردن رمز کاربریشان امکان بازیابی یا تغییر رمزشان را داشته باشند.
ساخت کاربر در بار نخست به عهده‌ی مدیر است ولی پس از آن بقیه امور مانند تعیین و تغییر رمز عبور با
خود کاربر خواهد بود.
}

\subsection {نیازمندی شماره ۶}
\subsubsection {تعیین پارامتر‌های گزارش}
\noindent {
کاربر باید بتواند مقادیری مانند دوره‌ی زمانی مورد نظر برای رکورد‌های ورودی به
سیستم و یا تعداد کل رکورد‌های
\lr{import}
شده را مشخص کند.
تعداد این پارامتر‌ها بستگی به نوع گزارش خواهد داشت.
}

\subsection {نیازمندی شماره ۷}
\subsubsection {امکان \lr{Export} کردن گزارش‌ها}
\noindent {
کاربر باید بتواند گزارش‌های خود را به قالب‌های رایج مثل
\lr{pdf, jpg, eps, svg, png} 
و
\lr{postscript}
ذخیره کند.
}

\section {نیازمندی‌های غیر عملیاتی}
\noindent {
در این قسمت نیازمندی‌های عملیاتی نرم‌افزار دایکر آورده شده اند.
}

\alertwarningbox {
در این نسخه از این سند نیازمندی غیر عملیاتی خاصی برای دایکر
ذکر نشده است.
}

\clearpage
\section {نمودارهای \lr{UML 2.0}}
\noindent {
این قسمت حاوی نمودار‌های
\lr{UML}
\LTRfootnote {Unified Modeling Language 2.0}
می‌باشد.

\subsection {نمودار‌های \lr{Use-Case}}

\begin{figure}[!ht]
  \centering 
  \includegraphics[width=0.9\textwidth]{figures/Usecase1}
  \caption[\lr{Daycher Use-Case \#1}]
  {در این نمودار نقش آفرینان سامانه و تعاملشان با دایکر نشان داده شده اند.}
\end{figure}

}

\section {فرایند توسعه و چرخه عرضه}
\subsection {فرایند توسعه}
\noindent {
برای توسعه‌ی دایکر از متودولوژی توسعه نرم‌افزاری چابک
\LTRfootnote{\lr{Agile Software Development}}
استفاده می‌شود.

فرایند توسعه شامل گام‌های کوتاه
\LTRfootnote{\lr{Iterations}}
توسعه نرم‌افزار است که هر گام منتهی به عرضه نرم‌افزار قابل استفاده می‌شود.
هر گام توسعه شامل فعالیت‌های
\LTRfootnote{\lr{Activities}}
متعارف در فرایند توسعه نرم‌افزار خواهد بود.
این فعالیت‌ها(ترتیب انجام درون هر گام مهم نیست) شامل
\lr{Requirement (re-)analysis, Design, Implementation, Test and Debug, Release}
و
\lr{Maintenance}
خواهند بود.
جزییات و ترتیب کلی انجام این فعالیت‌ها بر عهده‌ی تیم پیاده سازی خواهد بود.
}

\subsection {چرخه عرضه}
\noindent {
به صورت ایده‌آل در فرایند توسعه نرم‌افزاری چابک هر گام منتهی به عرضه نسخه‌ی قابل استفاده
نرم‌افزار تحت توسعه می‌شود لذا در واقع معمولا بیشتر از یک گام برای عرضه نرم افزاری با حداقل کارایی
نیاز است.
برای دایکر ۳ عرضه در نظر گرفته شده است. عرضه اول مربوط به حداقل خصوصیاتیست
\LTRfootnote{\lr{Features}}
که دایکر برای قابل استفاده بودن به آن‌ها نیاز دارد.
برای نمونه پیاده‌سازی
\lr{Dashboard}
گزارش‌ها.
عرضه دوم مربوط به خصوصیات مهم دایکر است که در صورت کمی وقت می‌توانیم از پیاده سازی و افزودن آن‌ها به دایکر
صرف نظر کنیم. برای مثال امکان اتصال به تعداد زیادی پایگاه داده برای دریافت اطلاعات خرید مشتریان.
عرضه سوم مربوط به خصوصیات باقی مانده از دو عرضه‌ی پیشین است.
در صورتی که خصوصیاتی در دو عرضه قبلی به هر دلیلی مانند کمی وقت و یا منابع دیگر
قابل افزوده شدن به دایکر نبوده باشد در این عرضه دایکر اضافه می‌شود.
}