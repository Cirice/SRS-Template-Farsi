\chapter{خصوصیات سامانه}
\clearpage

\noindent {
}

\section {نیازمندی‌های عملیاتی}
\noindent {
}

\section {نیازمندی‌های غیر عملیاتی}
\noindent {
}

\clearpage
\section {نمودارهای \lr{UML 2.0}}
\noindent {
این قسمت حاوی نمودار‌های
\lr{UML}
\LTRfootnote {Unified Modeling Language 2.0}
می‌باشد.

\subsection {نمودار‌های \lr{Use-Case}}
این قسمت حاوی نمودار‌های
\lr{Use-Case}
سیستم تحت توسعه است.
}

\section {فرایند توسعه و چرخه عرضه}
\subsection {فرایند توسعه}
\noindent {
برای توسعه‌ی دایکر از متودولوژی توسعه نرم‌افزاری چابک
\LTRfootnote{\lr{Agile Software Development}}
استفاده می‌شود.

فرایند توسعه شامل گام‌های کوتاه
\LTRfootnote{\lr{Iterations}}
توسعه نرم‌افزار است که هر گام منتهی به عرضه نرم‌افزار قابل استفاده می‌شود.
هر گام توسعه شامل فعالیت‌های
\LTRfootnote{\lr{Activities}}
متعارف در فرایند توسعه نرم‌افزار خواهد بود.
این فعالیت‌ها(ترتیب انجام درون هر گام مهم نیست) شامل
\lr{Requirement (re-)analysis, Design, Implementation, Test and Debug, Release}
و
\lr{Maintenance}
خواهند بود.
جزییات و ترتیب کلی انجام این فعالیت‌ها بر عهده‌ی تیم پیاده سازی خواهد بود.
}

\subsection {چرخه عرضه}
\noindent {
به صورت ایده‌آل در فرایند توسعه نرم‌افزاری چابک هر گام منتهی به عرضه نسخه‌ی قابل استفاده
نرم‌افزار تحت توسعه می‌شود لذا در واقع معمولا بیشتر از یک گام برای عرضه نرم افزاری با حداقل کارایی
نیاز است.
برای دایکر ۳ عرضه در نظر گرفته شده است. عرضه اول مربوط به حداقل خصوصیاتیست
\LTRfootnote{\lr{Features}}
که دایکر برای قابل استفاده بودن به آن‌ها نیاز دارد.
برای نمونه پیاده‌سازی
\lr{Dashboard}
گزارش‌ها.
عرضه دوم مربوط به خصوصیات مهم دایکر است که در صورت کمی وقت می‌توانیم از پیاده سازی و افزودن آن‌ها به دایکر
صرف نظر کنیم. برای مثال امکان اتصال به تعداد زیادی پایگاه داده برای دریافت اطلاعات خرید مشتریان.
عرضه سوم مربوط به خصوصیات باقی مانده از دو عرضه‌ی پیشین است.
در صورتی که خصوصیاتی در دو عرضه قبلی به هر دلیلی مانند کمی وقت و یا منابع دیگر
قابل افزوده شدن به دایکر نبوده باشد در این عرضه دایکر اضافه می‌شود.
}
